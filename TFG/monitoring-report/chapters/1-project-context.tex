\chapter{Project contextualization}

Environmental sustainability has been a key topic for the past few years.

This project aims to raise questions about our current relationship with
nature and our way of understanding it.

In our current world of fast and efficient communication, we can no longer
trust on old ways of transmitting
information for the ever-increasing technologically savvy population.

The world of today calls for a new form of thought-provoking projects and
manifestos, it is for this reason a mobile video game will be used for this
endeavor.

This project will consist of building a mobile game that teches its players 
about the dynamics that come into play when we interact with nature
and how can we establish a healty and sustainable relationship with it.

\section{State of the art}

There are numerous papers that work on the subject of serious
games about education for sustainability. The majority of the work on this
field relies on more traditional gaming experiences, such as board games
or role playing experiences in the classroom or via interviews with real life
stakeholders.

Several games in the mobile market loosely follow the tone of this project,
based on a keyword search on the Android Play Store, we can extract a
collection of games that are valuable to get inspiration from based on the
project's objectives.
Although a lot more games have been downloaded and played by the
Researcher, only a handful have been found to be related with the project's
objectives.

\section{Technologies used for this project}

Different Game Engines were considered for developing this project. Amongst
all possible solutions, the Unity game engine was chosen for this project.
The reasons for this choice are the flexibility and extensibility of the
engine, the affordable price for personal use (Free until the project receives
100 thousand dollars in funding or revenue) and its powerful capabilities for
multi-platform development.
