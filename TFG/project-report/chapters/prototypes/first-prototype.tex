\section{First prototype}

It was intended to be a game about Landscape management, much like 
SHRUB BATTLE \cite{Depigny2007}. A social component for the
game was also outlined.

Gathering inspiration from GREENIFY \cite{Lee2013}, it was thought that
a social component for the game would greatly benefit the overall impact
it had amongst several players. A mechanic where, upon completing
certain milestones, the player would be prompted with a screen asking
for them to share a screenshot of their "garden" to social media. This
mechanic would create a competition behaviour amongst the player's connections,
thus increasing the likelyhood of other players to download, play, and learn
from the game through competition.

\subsection{Idea}
The idea for the first prototype was basically managing a natural environment
by extending flora and fauna through it. The players would "own" a garden
they could fill with several types of plants, those plants providing
more resources for the player once planted. When a certain ammount of time
would pass, fauna would emerge from groups of plants and will reward
the player for caring for them.

The player would learn plantation dynamics, and how certain plants influence
certain behaviours through gameplay.

\subsection{Mechanics}
The player would move around their garden via an avatar they could interact
with by tapping and swiping with their fingers through the screen. In figure
\ref{fig:basic} we can observe the initial state of the game.

The players can also buy plants from the shop screen, as seen in
figure \ref{fig:shop}. Once a plant has been bought, they can
access their inventory (Figure \ref{fig:inventory}) to choose which
plant to place down (Figure \ref{fig:placing}).

Once a plant has been planted, the players must wait for
it to grow (Figure \ref{fig:growing}). One grown, it gives players a reward
\ref{fig:reward}).

Using this basic loop, players could build up interestingly-looking and
complex gardens, and share them amongst their firends.

\begin{figure}
    \centering
    \includegraphics[width=0.25\textwidth]{figures/prototype-1/basic}
    \caption{The player (center) surrounded by an empty "garden"}
    \label{fig:basic}
\end{figure}

\begin{figure}
    \centering
    \includegraphics[width=0.25\textwidth]{figures/prototype-1/shop}
    \caption{The shop window}
    \label{fig:shop}
\end{figure}

\begin{figure}
    \centering
    \includegraphics[width=0.25\textwidth]{figures/prototype-1/inventory}
    \caption{The inventory window. The window shows the plants owned by the
		player}
    \label{fig:inventory}
\end{figure}

\begin{figure}
    \centering
    \includegraphics[width=0.25\textwidth]{figures/prototype-1/placing}
    \caption{The player choosing wether to place the plant.
		The number above the plant shows the ammount bought. A confirmation
		button can be found below the plant. When the player presses it, the
		plant is placed.}
    \label{fig:placing}
\end{figure}

\begin{figure}
    \centering
    \includegraphics[width=0.25\textwidth]{figures/prototype-1/growing}
    \caption{A placed plant. The number shown below is the time left until
		the plant is grown.}
    \label{fig:growing}
\end{figure}

\begin{figure}
    \centering
    \includegraphics[width=0.25\textwidth]{figures/prototype-1/reward}
    \caption{A placed plant. The number shown above it is the reward ammount
		added to the player's resource pool}
    \label{fig:reward}
\end{figure}

\subsection{Teaching methods}

This prototype was intended to teach respect for nature through its 
awe-inducing beauty and equilibrium. No direct method of teaching
was going to be used, nor the player would be faced with "greener" alternatives
to its actions.

\subsection{Conclusions}
The idea was cancelled after building the prototype, as its mechanics were
considered too sublte to make effective teaching through it. The players
may had understood the underlying idea of the beauty in nature, but the
connection between their actions and its impact on it would be lost.
