\section{Second prototype}

The idea begind the second prototype was one of cooperation and social
dynamics behind product consumption and natural resource management. It
consisted of controlling several stakeholders inside the videogame to learn
about how each one's decisions impacted the entire chain of relations they
had one to another.

It was inspired by KARKONOSZE \cite{Krolikowska2007} because
of its approach to understand natural resource management
as a multi-process endeavour, and by MAE SALAEP \cite{Barnaud2007},
as the prototype also relied on findings brought up during a previous
gaming session to reconsider its players approach to the problem.

\subsection{Idea}

The main idea for this prototype was to explore the production chain through the lenses of various stakeholders in a fantasy setting. This fantasy setting
would be an island with a serious \gls{goblin} infestation.

The players would need to take action upon this problem by providing solutions to it.
Players would take the role of various stakeholders during the process
of ending the plage, finding the links between their actions and the repercusions
of them on the production chain.

Two forms of gameplay would have been drawn depending on their long term environmental sustainability. 

The unsustainable one would have been offered to the player as a first contact with the game,
the sustainable one being unlocked once the unsustainable one is completed,
as a way of providing the player with a contrasted version of what actions
have they used on their first run as opposed as to the second one.

There would have been three stakeholders the player could control
in this game:

\paragraph{Adventurer}
The Adventurer helps control the population of goblins in the area.

\paragraph{Shopkeeper}
The Shopkeeper buys and sells various items to and from the Adventurer and the Farmer. It also produces refined items.

\paragraph{Farmer}
Sows and harvests the land. Produces raw items from it.

Each one of them would perform different actions wether the game is trying
to make the player choose the unsustanable path or the sustanable one.

\paragraph{Game objectives: Unsustainable path}
The adventurer has to eliminate all goblins on the island because they attack the population. Once the goblin population dies out, the island would be safe, but the
ecosystem would suffer greatly.

\paragraph{Game objectives: Sustainable path}
The adventurer has to plant plenty of trees to help the forest grow again and keep goblins from attacking the village.
This means that the goblins get to live in peace, and the island retains
its ecosystem health.

\subsection{Mechanics}

The game flow and the mechanics vary greatly depending if the player is
playing through the sustainable path or the unsustainable one.

\subsubsection{Game loop: Unsustainable path}

The main focus of this path, as said previously, is for the Adventurer
to kill goblins. Each battle depletes the Adventurer's health,
so she has to fill it up with Health potions.

Health potions are made from Salamandertail, a ficticious ingredient
harvested from Salamanders. Each time an adventurer buys a health
potion, then, the whole process of obtaining it is reinforced.

Salamanders need to be fed Palmroot, a ficticious plant.

In this path, though, the names of the stakeholders are changed.
The Shopkeeper, given that she produces magical potions,
is the Magician. The Farmer, given that she gathers
resources from animals, is the Rancher.

\paragraph{Adventurer}
\begin{enumerate}
	\item The Adventurer kills Goblins.
	\item The Adventurer levels up if experience is enough.
	\item Goblins drop Gold.
	\item The Adventurer buys Health Potions from the Magician.
	\item The Adventurer uses Health Potions to tend his wounds.
	\item Go to 1 if the number of killed Goblins is < 100.
	\item The Adventurer has exterminated all goblins, the game is over.
\end{enumerate}

\paragraph{Magician}
\begin{enumerate}
	\item Buys Salamandertail from Rancher.
	\item Brews Health Potion from Salamandertail.
	\item Sells Health Potion to the Adventurer.
\end{enumerate}

\paragraph{Rancher}
\begin{enumerate}
	\item Plants Palmroot.
	\item Buys Salamander Eggs.
	\item Waits for the eggs to hatch.
	\item Waits for Palmroot to grow.
	\item Harvests Palmroot.
	\item Feeds Palmroot to Salamanders.
	\item Harvests Salamandertail from Salamanders.
	\item Sells Salamandertail to Shopkeeper.
\end{enumerate}

This path accelerates the deforestation of the island, and drives the goblins
further into the town, fact that causes the Adventurer to be increasingly
more brutal taking them down. goblins, in turn, get angrier at humans.

The path culminates with the destruction of the island's ecosystem,
then, the game asks its players if they are interested in other kind of
approach to the problem.

\subsubsection{Game loop: Sustainable path}

The main objective of this path is to control the goblin population using
non-violent tactics while rebuilding their natural ecosystem.

As a ways of controlling the population, the Adventurer must feed
Goblincake, a ficticious meal, to captured goblins before
releasing them back to nature. Goblins who are fed will take longer to
attack the town again.

Goblincake is baked from Goblinfruit, a fictitious fruit the Farmer produces.

It is the job of the Farmer to win the game this time, as it's her duty
to plant trees in her lot of land.

In this path, the Adventurer, given that her job is one of a more pacifist
kind, is dubbed the Ranger, and the Shopkeeper becomes the humble Cleric,
now selling hand-crafted pies to the Ranger.

\paragraph{Ranger}
\begin{enumerate}
	\item Buys Goblincake from Cleric
	\item Captures Goblins.
	\item Feeds captive Goblins with Goblincake.
	\item Releases Goblins to the wild again.
	\item Goblins drop Gold.
\end{enumerate}

\paragraph{Cleric}
\begin{enumerate}
	\item Buys Goblinfruit from Farmer.
	\item Bakes Goblincake from Goblinfruit.
	\item Sells Goblincake to the Ranger.
	\item Buys Treeseed from Farmer.
	\item Makes Treesprout from Treeseed.
	\item Sells Treesprout to the Ranger.
\end{enumerate}

\paragraph{Farmer}
\begin{enumerate}
	\item Plants Goblinfruit.
	\item Collects Goblinfruit.
	\item Sells Goblinfruit to Shopkeeper.
	\item Collects Treeseed from trees.
	\item Sells Treeseed to Shopkeeper.
	\item Buys Treesprout from Shopkeeper.
	\item Plants more trees.
	\item Go to 1 if the number of planted trees is < 100.
	\item Game won, Goblins will stop attacking the village because the forest is restored.
\end{enumerate}

This approach helps reduce the conflict and harm between humans and goblins,
an also helps convey the idea that wild animal "infestations" are a man-made
problem that calls for a efficient solution that attacks the root of the
problem.

The path finishes when enough trees are planted back, and the goblins
now have enough land to roam free.

\subsection{Teaching methods}
The game would teach players about social dynamics and consumer responsibility through its mechanics. The metaphor of this prototype calls for taking a deep
look onto society's already established loops and reflect upon how, as consumers,
interact with them.

\subsection{Conclusions}
The prototype was considered unadequate at the end of the designing phase,
considered too complicated and its design too rigid to organically convey
this project's objectives.
