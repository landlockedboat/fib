\section{Technical features}

In this section, the tecnical features implemented for the final
game will be explained in detail.

\subsection{Notification system}

Notifications are sent between C\# classes using a notification
system. Classes can subscribe to an event and, when the event is
triggered, all subscribed functions are called. This notification
sistem is implemented with function pointers.

\subsection{Town}

The town is an entity that controls all the occurrences in the village.
It holds references to each villager and building, aswell as to every
resource that's on the ground.

\subsubsection{Orders}

The town manages all Orders that the villagers have to perform.

An order is composed of a location and a type, aswell as an auxiliary
number to help keep track of resource ammounts the order manages.

Orders are stored on a \texttt{Queue} data structure.

As for references to other objects, they are stored in a generic 
\texttt{List} structure.

\subsubsection{Time management}

As to process the events as ingame days go on, a notification system is used.

A manager \gls{singleton}
class keeps track of time, and triggers a day change whenever the time for
it is adequate.

When the class triggers a day change, all callbacks subscribed to this
action type are triggered. This allows several classes to execute portions
of their code whenever a day change is triggered although the time manager
class does not "know" of their existance.

\subsection{Map and Pathfinding}


The Map is a simple Unity \gls{tilemap}.

The tiles in this map are programmed in a way that change their aspect
according to the number of neighbours of the same type as theirs have.

Pathfinding is done with the A* algorithm, and it operates over the tilmap.

The tilemap in of itself consists of various abstractions
and to fully integrate it with the game and the pathfinding, several transformations
have been executed.

The Unity Tilemap component is read by a \texttt{TilemapController} class,
that provides a useful API for the \texttt{NodeMap} class to read from. The
NodeMap is then encapsulated into a more useful \texttt{CellMapController}
that contains \texttt{Cell}s, that in turn encapsulate the \texttt{Node}s
of the previous class.

This layered approximation is done in order to not add overhead to the
pathfinding algorithm (the algorithm works with \texttt{Node}s, more
lean than its counterpart, the \texttt{Cell}) and helps the developers
build on top the already existing Tilemap component by using the
\texttt{TilemapController} class as a wrapper.

\subsubsection{Pathfinding}

\paragraph{The algorithm}

A* is a quite simple but relatively accurate and efficient way to find
paths in a connected graph.

The connected graph, in this case, is our \texttt{NodeMap}, and its connectedness
with neighbouring tiles depends on their \glspl{collider}. If a tile in the \texttt{NodeMap}
has a collider, its neighbouring tiles that do \emph{not}
hold a collider will be connected to it. 

On the contrary, if a tile does not have a collider, all its neighbours that,
like the previous one, do not have a collider, will be connected to it.

It could be useful for the reader to consider the presence or abscence of
colliders as a sign about wether the tile that has or does not have the
collider is walkable (does not have a collider) or it is not (it does have
a collider).

Considering a tile to be walkable means that it will be navigable by our
agent in this explanation of the A* algorithm.

The function for invoking the A* algorithm is given the starting
node and the destination node.

An open set of nodes $O$ is defined, containing the initial node.

At each iteration of tis main loop, the algorithm determines which of its
initial paths to expand. It picks the node $O$ that minimizes the $f$
function, that function being
\\
$f(n) = g(n) + h(n)$
\\
With $n$ being the node the calculation is being executed upon 
and $h$ being the heuristic function that calculates
the distance between $n$ and the target node.

In this implementation of the algorithm, the distance between two adjacent
non-diagonal nodes is 10, and the distance from a node to and adjacent
one that lies diagonally adjacent to it is 14. 

This approach is used in order
to approximate the value of $\sqrt{2}$ by multiplying it by 10 and rounding
it up. As the reader may already have intuited, the reason why $\sqrt{2}$ is
used to calculate distances between two diagonal nodes is because if
we interpret the nodes to be all 1 distance away from each other and to be
squares of 1 unit times 1 unit for its sides, a distance of $\sqrt{2}$ is,
effectively, the distance separating two adjacent diagonal nodes.

With this cleared up, we can continue on discussing the algorithm's
implementation.

If the chosen node $n$ happens to be the destination node, we simply report
the path and end the algorithm.

If this is not the case, we add $n$ to the closed set $C$ and start iterating
over its neighbours, i.e. the nodes adjacent to it.

If a neighbour $m$ happens to not be in $C$, we add it to $O$.

If $m$ happened to belong to $O$, we update its $g$ and $h$ cost and
not add it to $O$ again.

The new $g$ cost will be the distance from $n$ to $m$ (the addition of
all the nodes distances from the start to $n$ plus the distance from $n$
to $m$.

The new $h$ cost will be the distance from $m$ to the target node.

We delete $n$ from $O$. If no more nodes exist in $O$, the algorithm
has finished: no path has been found. On the contrary, the algorithm
loops again.

\subsection{Villager}

Villagers operate following a set of rules stablished by their AI.

Villagers ask for orders to the Town as long as they don't have any unattended
needs that need to be fullfilled.

Villagers break down those orders into several other phases, and then they
approach each one of them differenty according to its type.

An other has three phases: Initialisation, Transportation, Completition.

More complex orders are actually given to villagers by joining several orders.

For example, for a villager to put a piece of wheat in the oven, the villager
needs to first go to the warehouse and then to the oven. This is achieved
by enqueuing the order to feed the oven at the completition phase of its
predecessor.

Villagers use the pathfinding algorithm explained above to navigate the
island through the tilemap.

\subsection{Resources}

In the final game are several resources. Resources are every object that can
be dropped on the ground and picked up.

The town keep track of how many resources are stored in the warehouses at any
given time.

\subsection{Buildings and Factories}

Buildings are all objects that are static and cannot be picked.

Factories are special buildings that, given a resource and enough time,
they can transform it into another one.

\subsection{Level Configuration}

For tweaking each game's variables such as the seconds a day lasts
or the villager's walking speed,
a \gls{singleton} class has been used. Its values can be controlled
through Unity's Inspector, which makes it very easy to tweak.

\subsection{Serialization}

\Gls{serialization}
is done in order to keep player's progress stored across several play sessions.

Various data is being serialized in the game.

The map has to be serialized in order to not rebake the whole navigation
on top of it each time the game is loaded.

All the buildings, factories and villagers's positions and types are
serialized.

The ammount of resources stored in the town's warehouses is serialized 
aswell, as the number of days passed and several other metrics.

\subsection{Translations}

The final game con currently be played in two languages: Spanish and English.

The translations are stored in a .cvs file, and the games pick the text
to be shown to the player according to the language the player selected.

\subsection{Day/Night cycle}

The game operates in daily cycles. Because of that, a system to help emulate
a day and night cycle was developed to help the player get a better
grasp of the passing of days with a visual feedback.

The way this is done is by altering the values of the PostProcessingStack's
shaders that alter the final color composition of the image, turning it into
a blue color when the night is upon the village, and reverting
the image back to its inaltered state once daylight is again restored.

\subsection{UI components}

Several UI components have been developed for the game, the majority of
them being simple UI elements found in Unity by default such as the
Button or the ScrollField.

\subsection{ReWild console}

A language and a console to interpret it were developed to help on the
game's prototyping phase.

The language is a very simple one, cansisting of natural language phrases
to indicate a certain act to be performed in the game.

For example, a valid sentence in the ReWildConsole language would be:
\\
\texttt{build house 1 1}
\\

That means that a house must be built at the tile x = 1; y = 1.

\section{Cut out features}

\subsubsection{Logger}

The Logger was an utility that reported important happenings in the game world.

It worked in a way that text was shown on the bottom half of the screen,
a line for each happening.

It was cut out of the final game because it occupied a lot of screen space
and made having on-screen buttons very difficult.
