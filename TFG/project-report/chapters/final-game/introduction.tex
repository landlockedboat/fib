\section{Introduction}

The thesis for the final game is about animal agriculture and its impact
on planet earth. Animal agriculture is one of the biggest contributors
to \gls{GHG} emissions and climate change \cite{gerber2013tackling}.

Because this issue is widespread and easy to mitigate (consume more plant
based options that replace meat in some meals, replacing dairy with
plant based drinks, etc.) it was decided to be the main focus of the final
game and the project.
\subsection{Idea}

\subsubsection{Summary}

The final game is about managing a small islander town by building
the right structures in the right spots. It is a game about minimizing
commute times as much as optimizing space for letting the industries
operating in the community the space they will need.

Specifically, the final game will simulate two industries: the meat
industry and the tofu industry. The player will be able to assest
which one of them is more space-intensive and which one is not.

The meat industry will yield lower ammounts of food for the same
space needed for the tofu industry to yield more. Because of
this design decision it is hoped that the player will understand the
ecological benefits of reducing and/or replacing meat in meals.

\subsubsection{In depth}

The game is a town management simulator, with inspiration from other
games like Dwarf Fortress 
\footnote{\texttt{http://www.bay12games.com/dwarves/}},
Oxygen Not Included
\footnote{\texttt{https://www.klei.com/games/oxygen-not-included}},
and Rimworld
\footnote{\texttt{https://rimworldgame.com/}},
.

\paragraph{Story}

The main objective of the game is to finish all missions. The missions
involve building new buildings, surviving a certain ammount of days, etc.

The first missions revolve around setting up the basic town structure:
building homes, farms and ovens to produce bread, the first and most
basic edible in the game. The villagers will hold onto this diet until
the villagers start eating meat.

The missions become increasingly challenging as time goes on, up
until the player gets to the point where the players gets a special
mission, in which they are notified that their villagers no longer eat meat.

When the game reaches this point, the players will find out that it is 
actually easier for them to feed their villagers a plant based diet, as
it's more space efficient.

Some days after the last diet shift, the game will end and the player
will be prompted with a survey to fill up.

\paragraph{Gameplay}

The players does not interfere directly in the islanders life (the
town's inhabittants), they engage by placing orders for the villagers
to execute. The orders are set by building new buildings that the villagers
will engage with.

\paragraph{Buildings}

Each building, depending on its nature, creates certain orders that effect
the town and its villagers differently.

Buildings have also a price for each of them. Money is obtained
from the villagers each 3 days. Each villager adds between 25 and 45
resources to the resource pool.

Below there can be found a description of the buildings
and its prices.

\begin{itemize}
	\item \textbf{House} - A building where villagers can sleep.
		It costs 50 resources to build.
	\item \textbf{Warehouse} - Building that enables villager to store resources
		into. It costs 100 resources to build.
	\item \textbf{Dock} - Special building that makes possible new villagers
		coming to the town. It cannot be built.
	\item \textbf{Wheat farm} - A building that produces 1 wheat after a seed has been
		planted on it by a villager. It costs 25 resources to build.
	\item \textbf{Oven} - Building that transforms 1 wheat into 2 bread. It costs
		75 resources to build.
	\item \textbf{Pig pen} - Building that transforms 50 wheat into 1 pig.
		It costs 100 resources to build.
	\item \textbf{Slaughterhouse} - Building that transforms 1 pig into 25 meat.
		It costs 120 resources to build.
	\item \textbf{Soy farm} - A building that produces 1 soybean after a seed has been
		planted on it by a villager. It costs 25 resources to build.
	\item \textbf{Tofu fermenter} - Building that transforms 1 soybean into 1 tofu. It costs
		75 resources to build.
\end{itemize}

\paragraph{Art}

The art for the final game has been freely obtained from kenney's assets
\footnote{https://kenney.nl/}. They belong to the roguelike/RPG pack
\footnote{https://kenney.nl/assets/roguelike-rpg-pack} and to the UI
RPG expansion pack\footnote{https://kenney.nl/assets/ui-pack-rpg-expansion}.

\subsection{Mechanics}

Players engage with the game by placing buildings. They can buy
buildings (Figure \ref{fig:building-shop}), choose where to place
them (Figures \ref{fig:tap-to-build} and \ref{fig:place-building}) and
finally, placing them down (Figure \ref{fig:building-built}).

\begin{figure}
    \centering
    \includegraphics[width=0.25\textwidth]{figures/final-game/shop}
    \caption{The shop window}
    \label{fig:building-shop}
\end{figure}

\begin{figure}
    \centering
    \includegraphics[width=0.25\textwidth]{figures/final-game/tap-to-build}
    \caption{The player has to tap a part of the screen for the building to
		appear in}
    \label{fig:tap-to-build}
\end{figure}

\begin{figure}
    \centering
    \includegraphics[width=0.25\textwidth]{figures/final-game/place-building}
    \caption{The player can choose where to build the building. In green,
		a "blueprint" is being shown to the player for them to know
		where it will end up in. Also, buttons have been added
		to move the building more precisely}
    \label{fig:place-building}
\end{figure}

\begin{figure}
    \centering
    \includegraphics[width=0.25\textwidth]{figures/final-game/building-built}
    \caption{After the player chooses the location, the building can be seen
		placed in the desired position}
    \label{fig:building-built}
\end{figure}

\subsection{Teaching goals}

The final game's aim is to teach the player the correlation between a
community's eating behaviours and its impact on its sustainability.

As the game goes on, it becomes increasingly harder to feed all
the communitie's population off their initial meat based diet. As
the diet changes, the player will hopefully see the correlation between
their community's diet change and the change in farmland demand.
