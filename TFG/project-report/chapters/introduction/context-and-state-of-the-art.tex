\section{State of the art}

\subsection{Stakeholders}

\subsubsection{Researcher and Developer}

Those two functions are to be done by the thesis author.

The functions of the researcher during the project will be:

\begin{itemize}
\item Understanding the problem the project is trying to solve.
\item Search for information.
\item From a large stock of information, obtain the most relevant bits and extract the information in them.
\item From the most relevant information, find a solution to the original problem by examining it and creating an implementation specification.
\end{itemize}

The functions of the developer during the project will be:

\begin{itemize}
\item Understanding the researcher's specification of the implementation.
\item Work on the implementation following the specifications laid out by the researcher until the project is finished.
\item Send biweekly reports and a development build to the thesis supervisor during the project development.
\item Send development builds to the testers during development.
\item Correct bugs and adjust the implementation based on the feedback received from the testers and thesis supervisor.
\end{itemize}

\subsubsection{Testers}

Testers will mainly be the colleagues and classmates of the the thesis author and various other people that the author consider appropriate for this role.

Testers will:

\begin{itemize}
\item Play several minutes of the development build the developer has previously sent them.
\item Send feedback and bug reports to the developer based on the gaming experience. 
\end{itemize}

\subsubsection{Thesis supervisor}

The thesis supervisor for this project will be Professor Antonio Chica Calaf.

The functions of thesis supervisor during development will be:

\begin{itemize}
\item Oversee and control the project deadline
\item Make sure the objectives are completed
\item Help and give feedback to the thesis author during the length of this project.
\end{itemize}

\subsubsection{Activists and \glspl{NGO}}

Independent environmental activists and \glspl{NGO} may be interested in this project's conclusions, as those can be used as a tool for activism and/or for gaining reach and support. 

\subsubsection{Target audience}

The target audience for this game will be people with a capacity for understanding the dynamics of natural resources exploitation and the implications in the environment - and also has a smartphone.

This population segment will consist of a subset of mobile game consumers that pass the age of 8, as this age has been demonstrated
\cite{childdevelopment}
to be enough to grasp concepts as abstract as the one previously mentioned.

\subsection{Research on the subject}

There are numerous papers  that work on the subject of serious games about education for sustainability. The majority of the work on this field relies on more traditional gaming experiences, such as board games or role playing experiences in the classroom or via interviews with real life stakeholders.

During the course of the research process, several games have been analyzed and
will be briebly explained further below.

\subsubsection{From UNESCO's guide on simulation and gaming for environmental education
\cite{Taylor1983}}

\paragraph{Spring green motorway}
It is a roleplaying game about the building of a motorway through a village
and its outcomes on all stakeholders. It is thought of as a classroom
experience, with each student getting the role of a different stakeholder
in a sutuation of conflict regarding the building of a highway that runs
through the middle of a once quiet and pacific english village.

\paragraph{The houses game}
On a A4 sheet of paper, students must stick some previously built
paper houses to it following a certain criteria. The game aims to teach students
about the implications of urban planning and pushes them to understand new
criteria and ways to understand city layout design.

\paragraph{The poverty game}
It is a card game about poverty and disease and how they relate to farming
practices and the environment. The players take the role of an Central African
village and have to plan ahead the various inconveniences that may arise during
the season.

\paragraph{Caribbean fisherman}
This is a game about understanding the issues with exploiting the environment
and the problems it causes to workers and their families alike. This game
in particular focuses on fishing in a caribbean island.

\subsubsection{From Sage Publishing's Simulation and Gaming special
issue on Climate Change}

In this issue of Simulation and Gaming, connections between climate change
education and simulation, gaming and debriefing are drawn.

The issue highlight the importancy and effectiveness of raising awareness
about \gls{CC} via gaiming experiences.

The following paragraphs will consits of the game's title and a citation
to the paper it was originally developed in. Then, a brief description
of its focuses and teaching methods will be provided.

\paragraph{A Simulation of International Climate Regime Formation
\cite{Kauneckis2013}}
Present a classroom simulation that specifically addresses the
international relations dimension of global warming.

\paragraph{Keep Cool
\cite{Eisenack2013}}
Integrates issues of \gls{GHG} mitigation, climate-change adaptation, and global politics.

\paragraph{Greenify 
\cite{Lee2013}}
Game that uses social networks to promote actions against climate change.

\paragraph{A Perspective-Based Simulation Game to Explore Future Pathways of a Water-Society System Under Climate Change
\cite{Valkering2013}}
The case study is water management in
the Netherlands under future climatic conditions. The game dynamically maps the
involved actors’ norms and beliefs relative to water management within a scenario
exercise.

It is about conflicting points of view and is played with field professionals.

\paragraph{Land Policies for Climate-Change Adaptation in West Africa: A Multilevel, Companion-Modeling Approach
\cite{DAquino2013}}
The game presents another local-level adaptation study in West Africa.

\paragraph{If Local Weather Was Our Only Indicator: Modeling Length of Time to Majority Belief in Climate Change
\cite{Szafran2013}}
Focuses on the perception of climate change. The long-term, gradual process of climate change is difficult to apprehend for both individuals and communities because local, fast-fluctuating weather patterns vary from year to year.

\paragraph{Integrating Climate-Change Mechanics Into a Common-Pool Resource Game
\cite{Fennewald2013}}
This paper reflects on the challenges in the design of climate-change games.

\subsubsection{From Sage Publishing's Simulation and Gaming
issue on Natural Resource Management}

The following games' investigation focus is natural resource management.

Each game focuses on one or more resources and simulates one or more in game
stakeholders that have special interests on the resources being focused upon.

The following paragraphs will consits of the game's title and a citation
to the paper it was originally developed in. Then, a brief description
of its focuses and teaching methods will be provided.

\paragraph{SHRUB BATTLE
\cite{Depigny2007}}
Shrub battle is a game about understanding the development of vegetation in
rural areas, and how its development may affect the landscape and, in turn,
the touristic interest of the region, all while having in mind
the necessities of local farmers and ranchers.

\paragraph{KARKONOSZE
\cite{Krolikowska2007}}
This game explores the conflict over the exploitation of the Karkonosze mountains
in Poland. The skiing industry is rapidly taking over the natural resources
the mountains produce for its own benefit. This is a game about balancing
nature conservation efforts versus economic development in the region.

\paragraph{Microworld gaming of a local agricultural production chain in Poland
\cite{Martin2007}}
This game explores sustainable farming and Ecoagriculture.

\paragraph{BUTORSTAR
\cite{Mathevet2007}}
Butorstar explores the complexity of the Wetland ecosystem and its conservation
by focusing on the principal stakeholders that benefit from it. From herder to
scientists, the wetland is a very valuable ecosystem that affects animal
welfare and the livelyhoods of many people.

\paragraph{VPA-KERALA
\cite{Witteveen2007}}
In this paper, a game was developed to explore the topic of the management
of coastal zones.

The authors use video-filming techniques to create an interactive educational
tool, a \gls{VPA} on integrated coastal zone management in Kerala, southern
India.

First, students are given a fictional mission
and initial knowledge of the context by watching a documentary. Then they get the
opportunity to perform “virtual interviews” with local stakeholders based on 
audiovisual material. Finally, they explore the use of \gls{VPA}
directly with stakeholders and they discover that the mere projection
of these filmed interviews succeeds in enhancing social learning and
collective management principles.

\subsubsection{From other papers}

\paragraph{ATOLLGAME
\cite{Dray2006}}
A game about managing freshwater lenses, that is, water reserves built up
by rainfall below tropical islands, accessible through a well. This
lenses provide its own set of challenges when managing them, as their
overexploitation may lead to saltwater entering the wells because of the
wet limestone the lenses are found on top of. The game simulates the relations
between landowners, public officers and other stakeholders.

\paragraph{RIVER BASIN GAME
\cite{Lankford2004}}
A River basin management game, where the water is simulated using marbles
and a board on a slope to represent the river basin.
Players use small sticks to derive marbles for their own use just like
irrigators would do.

On the basis of several experiments in Africa, the authors discuss the ability
of this simulation game to generate new ways to efficiently manage this
resource.

\paragraph{PIEPLUE
\cite{Barreteau2007}}
Pieplue is about river basin management. It's a game that combines the use
of the \gls{RPG} and computerized simulation, as it facilitates the
exploration of a large range of time scales, the thesis authors explain.

The authors designed a test bed that
makes explicit the diversity of interactions among water users.
It also provides an
opportunity to establish a debate on water sharing amidst overall water use.

This game aim to explore the relationship between farmers and the water in
river basins.

\paragraph{DUBES, MÉÉRVISTIE, and others
\cite{Bots2007}}
A series of games about policy development. Players may take upon the
roles of stakeholders, clients, analysts and much more to explore the relations
of changing policies to a large ammount of natural resources.

Propose a six-dimension typology of functions associated with games in policy-oriented contexts.

\paragraph{JOGOMAN
\cite{Camargo2007}}
A game about water management in peri-urban areas. Players will explore the
various dynamics in urban development regarding to land and water usage by
taking upon the roles of various stakeholders. Architects, enterpreneurs,
public officers and tennants will have cooperate to find sustainable solutions
to their needs.

The authors point out that games provide a useful
learning environment to construct and share understanding about concepts,
procedures, and behavioral patterns.

\paragraph{FISHBANK-ILE
\cite{Qudrat-Ullah1997}}
A game about fisheries management and its delicate dependance on the natural
resources it depends on.

The author uses traditional experimental economics to simulate the problem
and proposes a formal approach to evaluate the benefit of a debriefing session after a
gaming experience.

In this paper, the authors deemed noteworthy that players who benefited
from knowledge exchanges tended to perform better than those
relying on their own judgment.

\paragraph{Surfing global change
\cite{Ahamer2006}}
This game allows one to walk through the complex argumentative landscape along changing roles and to identify societal consensus.

\paragraph{Water Wars
\cite{Hirsch2010}}
Game about water distribution in New Mexico.

\paragraph{MAE SALAEP
\cite{Barnaud2007}}
In its paper, the authors describe a series of simulation games involving
members of a community in Northern Thailand.
The paper's authors have developed a sequence of three different simulation
games representing the same area.

Each one of them deals with a specific issue
brought up during the previous gaming session.

The benefit of this iterative approach is
to allow stakeholders to deepen their understanding of the dynamics of natural
resource management while, at the same time, refining what aspect of it is more
important to them.

This games focus in watershed management. They provides a context to simulate
water and land management in watersheds and how farmers and other stakeholders
are affected by it.

% \paragraph{SIMS}
% In this research paper, the authors taught groups of students about
% responsible water and energy consumption by using the already developed and
% successful videogame The Sims.

\subsection{Video games on the market}

A variety of videogames have explored the subject of environmental sustainability.

Below there can be found a brief explanation of the basic mechanics of
some videogames.

\subsubsection{Mobile games}

Several games in the mobile market loosely follow the tone of this project, based on a keyword search on the Android Play Store, we can extract a collection of games that are valuable to get inspiration from based on the project's objectives.

Although a lot more games have been downloaded and played by the Researcher, only a handful have been found to be related with the project's objectives. Below it can be found a list of those games with an analysis of it in the context of the project.

\begin{description}

\item[Pixel Farm]{Farming simulator game played in \gls{portraitmode} with a \gls{pixelart}
aesthetic. Some mechanics about planting and harvesting seem interesting.}
\item[My Oasis]{Zen game about growing a garden in your mobile device. It is a \gls{clicker} game with \gls{lowpoly} 3D graphics. The player experiment a bonding experience with their tiny simulation, and can be a viable way to communicate with them.}

\item[Dont Starve]{It is a \gls{survivalgame} about resisting the harshness of nature and various monster attacks. Some mechanics like exploration and the dynamics of creature \glspl{npc} might be useful for the project.}

\item[Animal Crossing Pocket Camp]{A simplified version of the original Animal Crossing series for mobile devices. An industry giant with a very high ammount of polish. The general tone of the game generally is harmonious with nature, and its mechanics can be used to convey a sense of connection with nature}

\item[Desertopia]{An \gls{idle} \gls{clicker} game that consists of watering a desert to help nature in it sprout again. The general idea behind it aligns nicely with the objectives of the project and some ideas and mechanics could be used for it.}

\item[Pocket Plants]{A garden simulator game with a cartoon-y art style. Some of its mechanics and tone might be useful for gaming experiences oriented towards children.}

\item[Greening 2]{An \gls{idle} game about terraforming tiny planets by absorbing water and several other elements from asteroids and using them to fertilize the planet's soil. Very in depth with its mechanics and achievments, might be worthy to further explore for implementing a \gls{lategame} \gls{progressionsystem}.}

\item[Phone Story]{Phone Story is a game for smartphone devices that attempts to provoke a 
	critical reflection on its own technological platform.}
\end{description}

\subsubsection{Other platforms}

\paragraph{Eco}
Online MMO about crafting resources and coordinating to help
save the planet from climate crysis.

\paragraph{Climate Game}
Climate Game is an interactive online game that sets you on a quest to
settle on an uninhabited island covered by green trees and thick forests.

\paragraph{SimPachamama}
SimPachamama is an agent-based game to reduce deforestation and make
policy maintaining happy and developed society without biodegradation.


\paragraph{McDonald's Game}
Players indulge themselves in the complex process of managing a
supply chain in the food-focused corporation like McDonald’s.

\paragraph{Cities: Skylines}
Cities: Skylines is an interesting computer simulation game focusing 
on city building and management.

\subsection{Game Engines}

Different Game Engines were considered for developing this project. Amongst  all possible solutions, the Unity game engine was chosen for this project.

The reasons for this choice are the flexibility and extensibility of the engine, the affordable price for personal use (Free until the project receives 100 thousand dollars in funding or revenue) and its powerful capabilities for multi platform development.  

\subsection{Conclusions}

Despite extensive research in the subject, it is evident that not much of it has been applied to mobile gaming experiences. The project will be developed with one of the most popular and well-tested Game Engines in the market, and having in mind various mobile video games that are somewhat related to the project's objectives. All of this is done in order to ensure that the idea and the implementation of it on a mobile device is executed having in mind all the possibilities, both from the point of view of academic research and of mobile player-game interaction for obtaining interesting and applicable results.
