\section{Work done}

A working game has been built and published to the play store.
The game has been built over the course of 5 months. It contains 172
C\# scripts with over 8000 lines of code in total. A great effort has been
made to ensure the game would meet the requred personal standards the
thesis authors hold themselves worthy of.

Also, thanks to many playtersters
and players, enough data has been gathered about the effectiveness of
the game as a teaching tool.

The source code for the project can be found at 
\texttt{https://github.com/wextia/rewild/}.

\section{Interviewing results}

Over 20 individuals, ranging from the ages 18 to 27 have completed
the survey. If we recall section \ref{sec:player-interviewing}, 5
questions were given out to the players to fill up. The
results for this questions will be given now.

\begin{itemize}
\item I hold my experience with the game to have been an entertaining one.
\item I noticed that the game was finished, complete and free of bugs.
\item I understood the game's premise through its mechanics.
\item I think ReWild is a good metaphor for exploring the premise.
\item The game has made me reevaluate my relationship with nature and i'm considering taking action in response to it.
\end{itemize}

Players were asked to rate from 1 to 5, from a strong disagreement
with the question's statement and to a strong agreement of it.

For question 1, the median is 3; and the mean 3.65. 
The answers are pretty scattered in this one. Several players have
not had many fun with it, although a significant group of people rated
it a 5. This may be beacause players are biased towards not expressing
their honest thoughts to the thesis authors, despite the submission
process being anonymous.

For question 2, the median is 2 and the mean is 2.33. There were known
bugs to the developer at the beginning of the interviewing process,
and this has impacted the quality of the game very negatively. Time
constraints were a factor, and with a tight deadline polish was very
hard to achieve. Nontheless, many players finished the game without
encountering any game breaking bug and only noticing visual glitches.

For question 3, the median is 4 and the mean 4.35. This has been great
news for the thesis authors, as it makes the project's objectives
be reflected onto the game.

For question 4, the median is 4 and the mean 4.24. This only solidifies
what has been already noted in question 3.

For question 5, the median is 3, and the mean is 3.32. This question
has also had polarized results. Approximately 30\% of the interviewed
rated it 4 or higher, and roughly 40\% 2 or lower. Nontheless, the authors
are very pleased with the results, as it highlights that as much as
30\% of the players that played this game have understood its premise
and want to take action on it.

\section{Future work}

Although the final game yielded pretty positive results,
the thesis authors are certain about that some aspects of the game would
benefit from more work put onto those.

To help polish the experience up, some extensions for the work have been
proposed.

\begin{enumerate}
\item Adding more levels and variety to the game. This could be done in a
way that enables players to experiment with other harmful behaviours a 
community may engage with the natural resources that benefits from, such
as oil drilling, intensive farming, etc.
\item Adding sound effects would help players be more immersed in the
experience.
\item Adding real data in game to show the players the cost of their actions.
It would need to be done in a way that feels integrated with the game.
\end{enumerate}

The mentioned extensions are thought to be a good starting point for future
research on the subject.
