\chapter{Introduction}
\section{Formulation of the problem}

Environmental sustainability has been a key topic for the past few years. It has been demonstrated
\cite{stateofglobalair}
that our current level of consumption and, by consequence, resource-depletion has and is having a very harmful impact on planet Earth and its biosphere. Several measures are currently being taken
\cite{climatechangepolicies}
by corporations and governments alike, but our planet may be calling for a more fast and radical change in our way of perceiving nature.

This project aims to raise questions about our current relationship with nature and our way of understanding it. In our current world of fast and efficient communication, we can no longer trust on old ways of transmitting information
\cite{teensdontread}
for the ever-increasing technologically savvy population. The world of today calls for a new form of thought-provoking projects and manifestos, it is for this reason a mobile video game will be used for this endeavor.

\section{Objectives of the project}

The objectives of this project are

\begin{itemize}
	\item Study current teaching methods for educating people about environmental sustainability.
    \item Compare and choose the method that can be best transformed into a mobile gaming experience.
    \item Implement the method as a mobile game.
    \item Analyze its impact on players by a face-to-face post-play interview.
\end{itemize}

\section{Context}

The project is developed having in mind the existence of the Serious Games movement, that is, the movement of game developers creating games that aim to not (exclusively) entertain the player, but rather make them obtain an extrinsic reward through their game-playing experience.

Some examples of serious games are flight simulators, medical surgery practice games or the Animal Equality virtual slaughterhouse experience.
\footnote{http://ianimal360.com/}

This project is a game and it is being developed under the category of a Serious Game and bibliography about them will be cited thorough the length of the thesis.

\section{Stakeholders}

\subsection{Researcher and Developer}

Those two functions are to be done by the thesis author.

The functions of the researcher during the project will be:

\begin{itemize}
\item Understanding the problem the project is trying to solve.
\item Search for information.
\item From a large stock of information, obtain the most relevant bits and extract the information in them.
\item From the most relevant information, find a solution to the original problem by examining it and creating an implementation specification.
\end{itemize}

The functions of the developer during the project will be:

\begin{itemize}
\item Understanding the researcher's specification of the implementation.
\item Work on the implementation following the specifications laid out by the researcher until the project is finished.
\item Send biweekly reports and a development build to the thesis supervisor during the project development.
\item Send development builds to the testers during development.
\item Correct bugs and adjust the implementation based on the feedback received from the testers and thesis supervisor.
\end{itemize}

\subsection{Testers}

Testers will mainly be the colleagues and classmates of the the thesis author and various other people that the author consider appropriate for this role.

Testers will:

\begin{itemize}
\item Play several minutes of the development build the developer has previously sent them.
\item Send feedback and bug reports to the developer based on the gaming experience. 
\end{itemize}

\subsection{Thesis supervisor}

The thesis supervisor for this project will be Professor Antonio Chica Calaf.

The functions of thesis supervisor during development will be:

\begin{itemize}
\item Oversee and control the project deadline
\item Make sure the objectives are completed
\item Help and give feedback to the thesis author during the length of this project.
\end{itemize}

\subsection{Activists and \glspl{NGO}}

Independent environmental activists and \glspl{NGO} may be interested in this project's conclusions, as those can be used as a tool for activism and/or for gaining reach and support. 

\subsection{Target audience}

The target audience for this game will be people with a capacity for understanding the dynamics of natural resources exploitation and the implications in the environment - and also has a smartphone.

This population segment will consist of a subset of mobile game consumers that pass the age of 8, as this age has been demonstrated
\cite{childdevelopment}
to be enough to grasp concepts as abstract as the one previously mentioned.


