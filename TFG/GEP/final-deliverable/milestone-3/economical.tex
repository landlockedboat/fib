\chapter{Economical Management}\label{economical}

In this chapter, a breakdown of the various costs the project will have is detailed. This chapter is divided in various sections where each of them explore the budget from a different angle.

This first proposal for a budged will later be revised at the end of the project with the aim of correcting any possible deviation from the original plan.

\section{Human Resources}

The budget for the various personnel working on the project. As the project evolves, the ammount of hours each professional works on the project may vary.

\begin{center}
    \begin{tabular}{ | l | r | r | r | }
        \hline
        \textbf{Role} & \textbf{Hours} & \textbf{Hourly wage} & \textbf{Total} \\ 
        \hline
        \hline
        Project Manager & 70 & 50,00€ & 3500,00€ \\  
        Researcher & 60 & 35,00€ & 2100,00€ \\
        Software developer & 140 & 20,00€ & 2800,00€ \\
        Interviewer & 40 & 15,00€ & 600,00€ \\
        Quality Asssurance & 40 & 20,00€ & 800,00€ \\
        \hline
        \textbf{Total} &  &  & 9800,00€ \\      
        \hline
    \end{tabular}
\end{center}

\section{Hardware}

The price of every piece of hardware used during the duration of the project and a calculus for the amortization
\footnote{Price for using those products during the project.}
of each one.

\begin{center}
    \begin{tabular}{ | l | r | r | r | }
        \hline
        \textbf{Product} & \textbf{Price} &
        \textbf{Useful life} & \textbf{Amortization} \\ 
        \hline
        \hline
        OneOnePlus A5010 5T & 559,00€ & 3 yrs. & 69,88€ \\  
        Lenovo NB ideapad 110-15ISK & 609,59€ & 3 yrs. & 45,72€ \\ 
        \hline
        \textbf{Total} & & & 115,59€ \\      
        \hline
    \end{tabular}
\end{center}

\section{Software}

The price for the subscriptions or for the full programs used for this project. The price detailed always corresponds to a comercial license, in case it is desired for the project to be distributed and charged for.

\begin{center}
    \begin{tabular}{ | l | r | r | r | }
        \hline
        \textbf{Product} & \textbf{Price} &
        \textbf{Useful life} & \textbf{Amortization} \\ 
        \hline
        \hline
        Unity 3d & 125,00€ & 1 mo. & 563,00€ \\  
        Rider IDE & 349,00€ & 1 yrs. & 131,00€ \\
        Windows 10 & 145,00€ & 3 yrs. & \footnotemark
        0,00€\\ 
        Android SDK & 0,00€ & -- & 0,00€ \\ 
        iOS SDK & 0,00€ & -- & 0,00€ \\ 
        \hline
        \textbf{Total} & & & 693,00€ \\      
        \hline
    \end{tabular}
\end{center}
\footnotetext{Included in laptop price}

\section{Licensing}

Several other fees have to be payed in order to publish the final game on a mobile marketplace. Below it can be found the price for a developer license for both the Android and the iOS app stores.

\begin{center}
    \begin{tabular}{ | l | r | r | r | }
        \hline
        \textbf{Product} & \textbf{Price} &
        \textbf{Useful life} & \textbf{Amortization} \\ 
        \hline
        \hline
        Android developer license & 25,00€ & 10 yrs. & 1,00€ \\  
        iOS Developer Program & 75,68€ & 1 yrs. & 28,38€ \\
        \hline
        \textbf{Total} & & & 29,38€ \\      
        \hline
    \end{tabular}
\end{center}

\section{Indirect spending}

Several other spendings will be taken into account, mainly in form of utilities and office material. This budget is approximated using various sources
\footnote{https://comparadorluz.com/faq/precio-kwh-electricidad}
\footnote{https://comparadorluz.com/faq/precio-kWh-gas-natural}.

\begin{center}
    \begin{tabular}{ | l | r | r | r | }
        \hline
        \textbf{Product} & \textbf{Price per unit} &
        \textbf{Units} & \textbf{Total} \\ 
        \hline
        \hline
        Electricity & 0,12€/kWh & 1050kWh & 47,25€ \\  
        Gas & 0,05€/kWh & 2000kWh & 37,50€ \\  
        Internet & 65€/mo. & 4,5 mos. & 110,00€ \\  
        Office material & 32,50€ & 1 & 32,50€ \\  
        \hline
        \textbf{Total} & & & 226,95€ \\      
        \hline
    \end{tabular}
\end{center}

\section{Total budget}

Once all sections are budgeted, a final budget can be drawn from all of them.

\begin{center}
    \begin{tabular}{ | l | r | }
        \hline
        \textbf{Concept} & \textbf{Price} \\
        \hline
        \hline
        Human Resources & 9800,00€ \\  
        Hardware & 115,59€ \\  
        Software & 693,00€ \\  
        Licensing & 29,38€ \\
        Indirect spending & 226,95€ \\
        \hline
        \textbf{Total} & 10865,25€ \\      
        \hline
    \end{tabular}
\end{center}

\section{Conclusions}

It is possible that budget deviations occur if any piece of hardware needs repairing or schedule alterations occur. Hardware reparations will be reflected on the final draft of the budget (if any). Schedule alterations, although likely, are not very prone to destabilize the entirety of the budget, given that the total ammount of hours spent on the project will not vary much.  

It is important to have in mind that project is not aimed to generate revenue. The final game will be published as a free to play experience, and alternate sources of revenue will be explored to help mitigate the costs of the project. Contact with the goverment and several \glspl{NGO} will be key to help boost the project's funding.

It is possible that, given the final product will be usable for education, public funding is a attainable.
