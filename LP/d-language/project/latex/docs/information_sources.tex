\section{Information Sources} \label{sec:information_sources}

\subsection{Description of information sources}\label{subsec:description_of_information sources}

The majority of the informations sources come from the official D language webpage. Through all the article the reader can find several references to pages inside it.

A big part of the article is sourced in articles by software developers familiar with the language, with varying levels of involvement and renoun.

Few sections cite Wikipedia as a source.

Section \ref{subsec:programming_paradigms} is partly based on the D language book, wrote by one of the fathers of the language, Andrei Alexandrescu.

The entirety of section \ref{sec:other_characteristics} and its code examples are based on an article by Adam Hawkins, all code examples being of his making.

\subsection{Evaluation of information quality}\label{subsec:evaluation_of_information quality}

Overall, the information quality i find is very adecuate to the required level of the article, most claims being sourced on official documentation and/or articles by experimented engineers.

Some parts of the article i find could be better sourced, such as section \ref{subsec:programming_paradigms}, in which the list of supported paradigms is extracted from Wikipedia, and the page itself doesn't cite any other surces to back it up.

Another part in which the information may be biased is section \ref{sec:applications}. It is sourced on the official webpage, which may portrait the language in an exceedingly good way.

Finally, in section \ref{sec:similar_languages} the only source cited is a Wikipedia article, which in turn has no other citations regarding this topic inside it.

In conclusion, i find that even in those sections in where the information may be a bit dubious, the level of rigor required for this article is adequate, and the sources enough and varied given the nature and objective of this work, which is to give the reader a brief overlook of D and its more interesting functionalities before delving more deeply into specialised, technical documentation.
