\subsection{Experimento 4}

\paragraph{Enunciado}

Dado el escenario de los apartados anteriores, estudiad como evoluciona el 
tiempo de ejecución para hallar la solución para valores crecientes de los 
parámetros siguendo la proporción 4:100. Comenzad con 4 centros de datos e
incrementad el número de 2 en 2 hasta que se vea la tendencia. Usad el
algoritmo de Hill Climbing y la misma función heurística que antes.

\paragraph{Condiciones del experimento}

Para realizar este experimento necesitamos generar los datos correspondientes
a un total de 5 ejecuciones bajo diferentes criterios: 4 centros, 100 sensores;
6 centros, 150 sensores; 8 centros, 200 sensores; 10 centros, 250 sensores.

Para garantizar que nuestra solucion es correcta, realizaremos 10 ejecuciones
por cada caso y trabajaremos con su media aritmética. A parte, para obtener una
muestra representativa, generaremos las semillas para cada ejecución de manera
aleatoria.

Podemos resumir las características del experimento en la siguiente tabla:

\begin{tabular}{ | p{0.4\textwidth} | p{.6\textwidth} | }
  \hline
  Observación & El tiempo de ejecución puede variar aunque la proporción
  sensores-centros se mantenga constante\\
  \hline
  Planteamiento & Escogemos diferentes configuraciones de numeros de centros
  y sensores \\
  \hline
  Hipótesis & El tiempo de ejecución varía a pesar de que no cambie la 
  proporción de sensores-centros (H0) o sí.\\
  \hline
  Método & 
    \begin{itemize}
      \item Escogemos semillas aleatoriamente para todas las repeticiones
      \item Ejecutamos 10 experimentos para cada configuracion de centros
        sensores que necesitamos
      \item Usamos el algoritmo de HillClimbing 
      \item Mediremos el tiempo para realizar la comparación
    \end{itemize}
    \\
  \hline
\end{tabular}


\paragraph{Resultados del experimento}

\begin{figure}[h]
  \centering
  \includegraphics[scale=0.5]{exp4}
  \begin{tabular}{ | l | c | }
    \hline
    Número & Configuración del estado \\
    \hline
    1 & 4 centros y 100 sensores \\
    2 & 6 centros y 150 sensores \\
    3 & 8 centros y 200 sensores \\
    4 & 10 centros y 250 sensores \\
    \hline
  \end{tabular}
  \caption{Evolución del tiempo de ejecución bajo varias configuraciones
  y la leyenda apropiada}
  \label{fig:exp4-time-evolution}
\end{figure}

En la figura \ref{fig:exp4-time-evolution} podemos ver como el tiempo de
ejecucion varía en funcion de la configuracion que hemos utilizado.
Cada configuración va representada a partir de un número.

\paragraph{Conclusiones}

El resultado del experimento se asemeja mucho a lo que nosotros nos esperabamos
que sucediese según la hipótesis nula H0. El tiempo de ejecución va aumentando
a medida que aumentamos el numero de centros y sensores, de una manera casi
exponencial.

