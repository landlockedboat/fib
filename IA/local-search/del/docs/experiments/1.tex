\subsection{Experimento 1}

\paragraph{Enunciado}

Determinar qué conjunto de operadores da mejores resultados para una función 
heurística que optimice el criterio de calidad del problema (3.2) con un 
escenario en el que el número de centros de datos es 4
y el de sensores es 100. Deberéis usar el algoritmo de Hill Climbing. 

Escoged una de las estrategias de inicialización de entre las que proponéis. 
A partir de estos resultados deberéis fijar los operadores para
el resto de experimentos.

Pensad que con estas proporciones, se podrán transmitir todos los datos.

\paragraph{Conclusiones}

Como tan solo contamos con un operador, no tiene sentido plantearse la pregunta
de que conjunto de operadores es mejor\footnote{
  Cabe destacar que nuestro operador \texttt{connect} garantiza la
  restricción 3 (No se pueden establecer más de 25 conexiones a un 
  centro de datos o más de 3 a un sensor.) de la sección 3.2 Criterios
  de la solución.
}. 

\paragraph{}

