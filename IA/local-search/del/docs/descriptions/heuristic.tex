\subsection{Función heurística}

La función heurística es definida por esta fórmula:


\[ h(state) = state.all\_connections\_cost + state.lost\_data^2 \]

Donde $state$ representa el estado sobre el cual medimos la heurística,
$all\_connections\_cost$ representa el coste\footnote{
  Recordemos que el coste de una conextión entre dos sensores
  $x$ e $y$ es definido por la función 
  $coste(x, y) = d(x, y)^2 \times v(x)$ donde $d(x,y)$ es la distancia 
  euclídea que los separa
  y $v(x)$ es el volumen de datos que $x$ está transmitiendo a $y$.
}
de todas las conexiones activas en el estado y $lost\_data$ representa toda
la información que ha sido perdida en el estado.

\paragraph{}

Hemos escogido esta función porque creemos que representa correctamente
nuestras prioridades, dado que es una función que debemos minimizar.
Los datos perdidos están elevados al cuadrado
para compensar de alguna manera el factor cuadrático del cálculo del coste
de conexión (el coste asumimos que es realizado con la distancia al cuadrado
para no tener que realizar una operación de raíz cuadrada 
\texttt{Mathf.sqrt} para calcular la distancia euclídea real, ya que es muy
ineficiente).
