\documentclass{beamer}
\usepackage[spanish]{babel}
\graphicspath{ {figs/} }
\usepackage[utf8]{inputenc}
\setbeamertemplate{navigation symbols}{}

\AtBeginSection[]{
  \begin{frame}
  \vfill
  \centering
  \begin{beamercolorbox}[sep=8pt,center,shadow=true,rounded=true]{title}
    \usebeamerfont{title}\insertsectionhead\par%
  \end{beamercolorbox}
  \vfill
  \end{frame}
}

\title{The grid}
\author{Raluca Vijulie \and Martí Rosell \and Víctor Alcázar}
\institute{Inteligencia Artificial}

\date{2016}

\begin{document}

\frame{\titlepage}


\begin{frame}

\frametitle{Que es The grid}
\begin{itemize}
  \item<1-> Un servicio web que diseña y constrye paginas web
  \item<2-> El usuario introduce contenido y el programa lo estructura
  \item<3-> El algoritmo responsable combina el contenido con gusto
\end{itemize}

\end{frame}

\begin{frame}
  \frametitle{Ejemplos}
  \includegraphics[scale=.22]{figs/example1.png}
\end{frame}

\begin{frame}
  \frametitle{Ejemplos}
  \includegraphics[scale=.22]{figs/example2.png}
\end{frame}

\begin{frame}
  \frametitle{Ejemplos}
  \includegraphics[scale=.22]{figs/example3.png}
\end{frame}

\begin{frame}

  \frametitle{Algunas funcionalidades}
  \begin{itemize}
    \item<1-> Crea automáticamente paletas de colores a partir de imagenes
    \item<2-> Distribuye el contenido para repartir la cantidad de 
      informacion dada al usuario
    \item<3-> Detecta puntos de interes de las imagenes para no 
      recortarlas
  \end{itemize}

\end{frame}

\begin{frame}
  \frametitle{Implementacion}
    \begin{itemize}
      \item<1-> The grid funciona con arboles decisionales
      \item<2-> La técnica es la mejor para un sistema con aprendizaje 
        supervisado
      \item<3-> Funcionan extrayendo informacio del usuario y llegando a 
        una conclusion final
      \item<4-> Ejemplo: \texttt{akinator.com}
    \end{itemize}
\end{frame}

\begin{frame}
  \frametitle{Implementacion}
    \begin{itemize}
      \item<1-> El usuario responde una serie de preguntas inicial
      \item<2-> Los arboles son construidos y extendidos con
        ayuda de diseñadores graficos
      \item<3-> Funcionan infiriendo respuestas del entorno
    \end{itemize}
\end{frame}

\begin{frame}
  \frametitle{Algunas criticas}
    \begin{itemize}
      \item<1-> El codigo generado es lento (javascript para estructurar)
      \item<2-> El algoritmo no tiene en cuenta la semantica del contenido
      \item<3-> Es poco flexible, ya que solo genera ``feeds'' al estilo facebook
    \end{itemize}
\end{frame}

\begin{frame}
  \frametitle{Rentabilidad}
    \begin{itemize}
      \item<1-> The grid es una startup que ha recaudado 6M\$ de capital
      \item<2-> Ofrece un plan de subscripcion basico de 90\$ al año
      \item<3-> La empresa continua expandiendose y creciendo
    \end{itemize}
\end{frame}

\begin{frame}
  \frametitle{Impacto}
    \begin{itemize}
      \item<1-> Es un servicio que pretende eliminar gran parte del 
        trabajo de diseño y desarrollo web
      \item<2-> Hay bastante polemica entre los diseñadores web sobre
        si la pagina realmente funciona
      \item<3-> Es beneficioso para gente principiante
    \end{itemize}
\end{frame}

\end{document}
