\section{Descripción de cómo han sido utilizadas las técnicas}
\label{sec:desc_util}

A pesar de que no hay mucha información al respecto, podemos hacernos
una idea aproximada de cómo funcionan los árboles decisionales 
( explicados en la seccion \ref{sec:desc_tecnica} ) en \texttt{thegrid.io}.

\texttt{thegrid.io} opera preguntando al usuario un serie de preguntas
iniciales, que por orden son \cite{website_ai}:

\begin{description}
    \item[Branding] El usuario decide en cuanta medida el sitio web deberia
    ser estructurado por el contenido y cuanto por sus elecciones.
    \item[Color] El usuario puede escoger una paleta concreta de colores 
    para su web.
    \item[Layout] El usuario puede definir, a rasgos generales, si prefiere
    un web de contenido denso, minimalista o más orientada al arte y diseño.
    \item[Type] Al usuario se le da a escoger entre una serie de 
    tipografías para su sitio web.
\end{description}

Hemos de suponer que los árboles decisionales del algoritmo de 
\texttt{thegrid.io} operan con
input "inferido" a través de las preferencias y el contenido que el 
usuario ha indicado que debería aparecer en su página. Teniendo en cuenta
que la cantidad de información que el usuario da al sistema es, 
como acabamos de ver, mínima, podemos concluir que el feedback del usuario 
tiene un
impacto reducido en el producto final y que el ritmo del algoritmo es
dictado mayormente por el contenido, aunque parezca paradójico dado que 
estamos hablando de árboles de decisión.

Pensamos que el algoritmo opera a varios niveles de abstracción, dada
la naturaleza del diseño web. Es muy probable de que se trate, por lo tanto,
de una aglomeración de árboles comunicándose entre ellos más que uno solo.

Podemos asumir, por ejemplo, que un árbol se encarga de distribuir la carga
de la información sobre la página, generando así picos y valles de 
información que hacen que la experiencia de visualizado sea más amena.
Es posible que a otro nivel operen árboles decisionales que se 
encargan de estructurar cada una de las piezas de contenido por separado,
y después se comuniquen con árboles de niveles superiores para cuadrar
colores y situarlos en posiciones óptimas para la experiencia del usuario.

\texttt{thegrid.io} está creada para que los diseñadores profesionales
puedan ampliar la capacidad del algoritmo de construcción de páginas, lo
que nos lleva a asumir que la mayoría de decisiones que se toman durante
el procesado de una página son derivadas del conocimiento experto de los
diseñadores que han trabajado en el algoritmo. Un ejemplo de ejecución
sería el siguiente:

\begin{enumerate}
  \item ¿Es bueno este color de texto para esta imagen?
  \item 
  \begin{enumerate}
    \item Opacidad $\rightarrow$ Correcta  
    \item Combinacion con el color de la imagen $\rightarrow$ Correcta  
    \item Cohesión con el tono de la imagen $\rightarrow$ Correcta  
    \item \textbf{Veredicto} $\rightarrow$ Correcto
  \end{enumerate}
  \item ¿El texto debería estar centrado?
  \item $\cdots$
\end{enumerate}

En resumen, \texttt{thegrid.io} funciona mediante una variación de el 
algoritmo de árboles de decisión, en el cual el conocimiento es extraído
de una base de conocimiento creada por varios diseñadores que en su
mayor parte se constituye de restricciones y buenas prácticas de diseño
que se han aplicado al desarrollo web tradicionalmente durante los 
últimos años.

Muchos críticos al sistema señalan la robustez sobre el cual está 
construido \cite{the_grid_better}, y rechazan la idea de que este servicio o cualquiera pueda
desbancar a los artistas y diseñadores web en su totalidad. Sin duda es
un punto débil del algoritmo a tener en cuenta, aunque opinamos que todavía
es pronto para forjar un veredicto tan severo como el que algunos se han
atrevido a pronunciar.
