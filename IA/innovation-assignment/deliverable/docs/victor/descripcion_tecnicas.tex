\section{Descripción de las técnicas de IA que se han utilizado} 
\label{sec:desc_tecnica}

La técnica que se ha utilizado es la de arboles decisionales 
 \cite{10_startups}.

La técnica de árboles decisionales es la mejor para un sistema que 
requiere de aprendizaje supervisado, como seria el caso de
\texttt{thegrid.io}, que tiene que adaptarse al feedback de los múltiples
diseñadores web que crean y modifican las reglas que regulan el algoritmo.

Un árbol decisional es una estructura de datos organizada 
jerárquicamente como árbol donde cada uno de sus nodos internos está
etiquetado con una variable de input y cada hoja está etiquetada por una
clase (resultado) que es devuelta cuando el algoritmo llega a ella.

Este arbol nos permite generar un algoritmo que lo recorra preguntando
al usuario en cada nodo interno por el valor de la variable de input de
él mismo. Según el input del usuario, el control del programa sigue por
una rama u otra del arbol. Este sistema nos permite modelar un algoritmo
basado en preguntas y respuestas que lo separa de los otros algoritmos
de Inteligencia Artificial.

El árbol de preguntas resultante nos sirve de lenguaje de representación, ya
que modela el conocimiento de nuestro sistema jerárquicamente.

Esta aproximacion a la resolucion de problemas tiene la ventaja de que
es fácilmente extensible. Si quisiéramos ensanchar la capacidad de 
resolución de nuestro algoritmo, sería muy sencillo: tan solo deberíamos 
añadir a un nodo hoja un subárbol decisional y repetir esta operación para
cuantos nodos hoja precisemos. De esta manera podemos expandir
e algoritmo de manera orgánica son afectar el desarrollo correcto de otras
ramas del mismo. Como ya he dicho antes, esto resulta extremadamente útil
para casos como el de \texttt{thegrid.io}.
