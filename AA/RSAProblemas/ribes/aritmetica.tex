\documentclass[a4paper,10pt]{article}
\usepackage[utf8]{inputenc}
\usepackage[catalan]{babel}
\usepackage{algorithmic}
\usepackage{dsfont}
%opening
\title{Exercicis Aritmètica Modular i RSA}
\author{
Albert Ribes Marzá
\and
Víctor Alcázar
\and
Kosmas Palios
}
\begin{document}

\maketitle

\begin{abstract}

\end{abstract}

% \section{}
\begin{enumerate}
 \item Calculeu:
 \begin{enumerate}
  \item $3^{28} \textrm{ mod } 10$
  
  \begin{equation}
   3^{2 \times 14} \textrm{ mod } 10
  \end{equation}
  
  \begin{equation}
   (3^2)^{14} \textrm{ mod } 10
  \end{equation}
  
  \begin{equation}
   9^{14} \textrm{ mod } 10
  \end{equation}
  
  \begin{equation}
   (-1)^{14} \textrm{ mod } 10
  \end{equation}
  
  \begin{equation}
   1 \textrm{ mod } 10
  \end{equation}
  
 

  
  
  \item $3^{200} \textrm {mod } 15$
  
    
  \begin{equation}
  3^{200}\ \textrm{mod } 15
  \end{equation}
  
    \begin{equation}
  3^4 \times 50\ \textrm{mod } 15
  \end{equation}
  
      \begin{equation}
  81^{50}\ \textrm{mod } 15
  \end{equation}
  
    \begin{equation}
  6^{50}\ \textrm{mod } 15
  \end{equation}
Com que $6^2\ \textrm{mod } 15  = 6$, el resultat és:
\begin{equation}
  6\ \textrm{mod } 15
  \end{equation}
 \end{enumerate}
\item \textbf{2EXP modular. }Doneu un algorisme de temps polinòmic que amb entrada els enters $a,b,c$ i un nombre primer $p$ computi $a^{b^c} \textrm{mod } p$

We first compute $x=b^c mod \phi(p)$ \\
Then we compute  $a^x mod p$.

Why is this correct? Simply because for every p,a, as Euler tells us that$a^{\phi(p)}=1 mod p$, we have the following $ a^k = a^{k+\phi(p)*l} mod p $.

Time complexity is polynomial, as modular exponention using repeated squaring is polynomial.

\item \textbf{Factorial Modular. }Donats dos enters $x$ i $N$, calcula $x! \textrm{ mod }N$.

 \begin{enumerate}
  \item Demostreu que un enter $y$ és primer si i només si per tot enter $x < y$ es compleix que $gdc(x!,y) = 1$.
  
  Hem de demostrar que:
  \begin{equation}
   y \textrm{ primer } \Leftrightarrow \forall x < y : \textrm{gcd(x!,y)} = 1
  \end{equation}
  
  La demostració té dos apartats:
  \begin{itemize}
   \item $\Rightarrow$
   
   EL gcd de qualsevol nombre primer sempre és 1 per un nombre primer i qualsevol altre, i es imposible que $x! == y$ ja que y no té factors. Queda demostrat
   
   \item $\Leftarrow$
   
   Ara demostrarem que: \\
   $y \textrm{ primer } \Leftarrow \forall x < y : \textrm{gcd(x!,y)} = 1$ \\
   
   Ho farem amb el contrarrecíproc: \\
   $ y \textrm{ compost } \Rightarrow \exists x < y : \textrm{gcd(x!,y)} \neq 1$ \\
   Com que $y$ és compost, existeixen dos naturals tals que $y = ab$, i es compleix que tots dos són menors que $y$.
   
   Llavors ja hem trobat un nombre menor que $y$ que compleix la condició: a \\
   $d = \textrm{gcd(a!, ab)}$ \\
   Està clar que a divideix d i que a no és 1, per tant d no pot ser 1. Queda demostrat.
   
  \end{itemize}


  
  \item Considera l'apartat previ per demostrar que si \textbf{Factorial Modular} fos computable en temps polinòmic, aleshores el problema de \textbf{Factoritzar} també sería computable en temps polinòmic (Recordeu \textbf{Factoritzar}: Donat un nombre enter $x$, calcula els seus factors primers).
 \end{enumerate}

 \item En un sistema criptogràfic \textbf{RSA} amb $p = 7$ i $q = 11$, troba la clau pública $(N,c)$ i la clau privada $(N,d)$ apropiades.

 Els passos per trobar les claus RSA amb dos primers $p$ i $q$ són:
 \begin{itemize}
  \item Computar $N = pq$
  \item Computar $\phi(N) = (p - 1)(q - 1)$
  \item Escollir $c \in \mathds{Z}_{\phi(N)}^*$
  \item Computar d tal que $cd \equiv 1 $ mod $\phi$(N)
  \item La clau pública és $P_B = (c,N)$ i la clau privada és $P_S = (d,N)$
 \end{itemize}
 
 Llavors els passos que fem són:
 \begin{itemize}
 \item $N = 7 \cdot 11 = 77$
 \item $\phi(N) = (7 - 1)(11 - 1) = 6 \cdot 10 = 60$
  \item Podem escollir entre moltes parelles per a $c$ i $d$. Algunes de elles son $(7,43)$, 
  $(11,11)$, 
  $(13,37)$, 
  $(17,53)$, 
  $(19,19)$, 
  $(59,59)$, 
  $(23,47)$, 
  $(27,47)$, 
  $(7,1)$. Nosaltres escollim 17 i 55
  \item Llavors les claus són: $P_B = (17, 77)$ i $S_B = (55, 77)$
 \end{itemize}


 
 
 \item \textbf{Sistema criptogràfic segur?. }Suposem que en lloc d'utilitzar un nombre compost $N = pq$ com es fa en el sistema RSA, utilitzem un nombre primer $p$. Per encriptar un missatge $m \textrm{ mod } p$ farem servir un exponent $e$, de la mateixa manera que es fa en el sistema RSA. L'encriptament del missatge $m \textrm{ mod } p$ seria $m^e \textrm{ mod } p$.
 
 Demostreu que aquest nou sistema no és segur donant un algorisme eficient per desencriptar. És a dir, doneu un algorisme que, amb entrada $p,e,m^c \textrm{ mod } p$, computi $m \textrm{ mod } p$ eficientment. Justifica la correctesa de l'algorisme i analitza el seu temps de computació.
 
 \textbf{La resposta comença aqui}
 
 In the given cryptosystem, to decrypt a ciphertext we must use the private key d, in the following way $m = c^d mod p$. The problem here is that the private key can be easily computed. We know that the relation between e and d is $ed = 1 mod \phi(p)$.
But now $\phi(p)=p-1$, because p is prime! 

So all we have to do is compute the multiplicative inverse of e modulo p-1. This can be easily done using the Extented Euclidean Algorithm, which takes $O(log(e)2)$ time. Then it is only a matter of exponentiation of c to the power of d modulo p. This in itself takes log(d) steps if done with repeated squaring. 

In total we have $O(log(e)2+log(d))$ time.
\end{enumerate}

\end{document}

