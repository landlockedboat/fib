\subsection{Quines reticències en mode de preguntes es planteja el professor
Joaquim Muns a l’hora de perdonar el Deute del 3 Mon?}

Joaquim planteja una serie de dubtes sobre si es bona idea perdonar els
deutes dels països pobres, enumerades a continuació:

\begin{enumerate}
  \item Els països pobres realment no poden pagar el deute? Hi ha algun
  problema sobre com es distibueix l'ajut als països del tercer món?
  L'ajut realment arriba al païs o es queda inutilitzat per motius de
  conflites armats?

  \item Molt lligat amb la primera: realment són els ciutadans els que 
  notarien aquest perdonament del deute o serien les èlits corruptes que
  han abocat el seu país a la situació actual les que les rebrien?

  \item Qui perdona aquest deute amb els països pobres? Realment és el
  país el que perdona el deute a un altre o són els ciutadans a través
  dels seus impostos?

  \item S'ha de perdonar el deute incondicionalment? La situació
  econòmicament precària de molts països en vies de desenvolupament
  és provocada per la corrupció i la injustícia. Si perdonem el deute,
  estem convidant a que aquesta situació es repeteixi?

  \item Per què no tractem el deute entre països igual que el deute entre
  persones? Al cap i a la fi, són els nostres diners els que s'han prestat
  a un altre país i hauríem de ser nosaltres els que ens preocupem de que 
  aquests diners es retornessin.
\end{enumerate}

