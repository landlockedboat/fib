\subsection{A què es refereix el professor Antón Costas quan parla de la 
``trampa de la Deuda''?}

Es refereix a que les mesures d'austeritat impulsades pels governs europeus 
en general i l'espanyol en concret son contraproduents per reduïr el deute
del pais en funció del seu PIB. És senzill veure per que:

Si els governs rebaixen els sous per augmentar la competivitat del mercat
o retallen despesa pública, el PIB del país disminuirà. Òbviament si mirem
la formula del PIB en funció del deute ens donarem compte que si disminuïm
el PIB del país, la relació del deute d'un pais sobre el seu PIB augmenta.

\[ fraccio = \frac{deute}{PIB} \]
