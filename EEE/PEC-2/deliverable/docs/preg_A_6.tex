\subsection{Suposem que na Carlota jugant a Borsa (rendiments del capital) 
ha tingut uns rendiments nets (abans d’impostos) de 34.000€ i 
que en Florenci, fent de taxista, també ha obtingut uns rendiments 
nets (abans d’impostos) de 34.000 €.
Quin dels dos tindrà la renda disponible més alta?}

Com que ambdues persones cobren el mateix, només ens queda analitzar
el que han de pagar d'impostos pels seus beneficis.

Els ingressos per rendiment de capital es paguen al 
19\%\cite{eleconomista_rendimiento_capital}, mentre que els taxistes 
han de pagar un 22,5\%\cite{sii_preguntas_frecuentes} d'impostos sobre 
la seva 
Renda Presumpta. Així doncs, el Florenci es quedaria amb una renda 
disponible de 26.350€ i la Carlota amb una de 27.540€, el que suposa que 
Carlota té una renda disponible més alta respecte al seu benefici net.
