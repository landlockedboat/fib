\section {
  La funció d’oferta mostra que quantes més unitats s’ofereixen, més
  augmenten els costos de fer el producte.
  Però l’idea general que té la gent és que
  quan comprem més unitats el preu és més baix. 
  Com s’explica aquesta contradicció?
}

És una resposta una mica complicada. La funció d'oferta mostra que el preu
d'elaboració dels productes augmenta amb la oferta perquè és la quantitat
de material que una empresa és disposada a produïr per un preu P.

Si una empresa no produeix més quantitat d'un producte és perquè no li
interessa vendre-ho a preu P. A la figura \ref{fig:offer-curve} podem
veure la funció oferta: el preu associat a una quantitat de producte.


\begin{figure}[h]
\caption{Funció de la oferta} 
\centering
\includegraphics[width=0.5\textwidth]{offer-curve}
\label{fig:offer-curve}
\end{figure}

