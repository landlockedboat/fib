\section {
  Explica breument com pot ser que l’oferta monetària sigui molt més
  gran que la Base Monetària. Els bancs ens estafen?
}

La oferta monetària és més gran perquè es tracta de "guanys" que el banc
hauria de generar a través d'inversions i prèstecs, però no es poden
mainfestar físicament, ja que encara no han estat cobrats. 

Òbviament això està molt regulat per la llei nacional i el FMI, entre
d'altres. Per llei, hi ha d'haver un mínim percentatge de diners físics
als bancs d'arreu del món (encara que sovint és una xifra bastant baixa).
