\section {
  Qué és el que marca el valor d’una moneda respecte a les altres?
}

La potència econòmica del païs que la produeix. Als anys 70, Nixon
va desenganxar el dòlar de l'estàndard or, el que va produïr que la majoria
de monedes també abandonés l'estàndard i l'enganxés al dòlar americà.

Això va suposar que la moneda té un valor molt relatiu. Normalment està
relacionat amb la productivitat del país i el seu número d'habitants, però
és molt difícil de determinar i sovint és per motius subjectius.

