\section{Elegeix un model d’estructura organitzativa apropiada i dibuixa l’organigrama. Explica els teus motius per a aquesta elecció.}

He escollit el model \textbf{matricial} d'estructura d'organització per a la empresa, interpretant cada punt de venda o producció com un \textit{projecte} individual amb un \textit{project manager} assignat a ell. Amb raó de complementar el gràfic he de destacar que:

\begin{itemize}
  \item El venedor encarregat és \textit{project manager} de cada botiga, per això està en negreta.
  \item El Director de Compres, Producció i Logística és el \textit{project manager} de l'obrador i de les oficines centrals, a falta d'un \textit{middle-manager} encarregat d'això.
  \item Cada persona dins la taula (excepte els \textit{project managers} i els directius) té dos directius als que ha de respondre: el seu \textit{project manager} i un cap de departament. Per exemple, el motorista de la botiga de Rubí té com a \textit{project manager} el venedor encarregat de la botiga i, com a cap de departament, al director de compres, producció i logística.
\end{itemize}

% For centering columns
\newcolumntype{P}[1]{>{\centering\arraybackslash}p{#1}}

\begin{center}
\begin{tabular}{ p{3cm}  P{3cm}  P{3cm}  P{3cm} }
  & \multicolumn{3}{c}{Directora general i de finances} \\
  \cline{2-4}
  & $\downarrow$ & $\downarrow$ & $\downarrow$ \\
  % Header
  &  Dir. Comercial & Dir. CPiL\protect\footnotemark & Dir. RRHH \\
  & $\downarrow$ & $\downarrow$ & $\downarrow$ \\
  % Body
  \hline
  \hline
  Obrador i Oficines & & Transportistes i Administratius & Operaris i Teleoperadors \\
  \hline
  \hline
  Botiga de Rubí & \textbf{Venedor encarregat} & Motorista & Venedors \\
  \hline
  \hline
  Botiga de Terrassa & \textbf{Venedor encarregat} & Motorista & Venedors \\
  \hline
  \hline
  Botiga de Sabadell & \textbf{Venedor encarregat} & Motorista & Venedors \\
  \hline
  \hline
  Botiga de Martorell & \textbf{Venedor encarregat} & Motorista & Venedors \\
  \hline
  \hline
  Botiga de St. Cugat del Vallès & \textbf{Venedor encarregat} & Motorista & Venedors \\
  \hline
  \hline
\end{tabular}
\end{center}
\footnotetext{Compres, Producció i Logística}
\addtocounter{footnote}{1}





La organització matricial permet una descentralització de les responsabilitats (el que va bé per un model de distribució basat en botigues) i al mateix temps permet que cada departament de l'empresa pugui prendre decisions a gran escala que afecten el seu departament sense cap mena de barreres burocràtiques. El model matricial genera una organització àgil durant el dia a dia (ja que els responsables de vendes de cada botiga prenen decisions sobre el que passa a les botigues diàriament) i genera una flexibilitat per a canvis a gran escala (redistribució del personal a les botigues, manteniment dels vehicles, etc) gràcies a que cada director té accés directe als empleats que treballen als seus camps d'especialitat.

El model clàssic, que es troba sota d'aquest paràgraf, genera una innecessària jerarquia que sobrecarrega les decisions d'uns directors i en relega a altres a llocs d' assessorament sense capacitat executiva directa (En aquest cas, recursos humans).

\begin{forest}
  for tree={
    draw,
    align=center
  },
  forked edges,
  [Directora general i de finances
    [Director comercial
      [Venedor encarregat, l=2cm,
      edge label={node[yshift=-5mm,midway,fill=white]{
        Per cada botiga
      }}
        [Venedors]
        [Motorista]
      ]
    ]
    [{Director de compres,\\ producció i logística}
      [Transportistes]
      [Operaris]
      [Administratius]
    ]
    [Directora de RRHH]
  ]
\end{forest}
