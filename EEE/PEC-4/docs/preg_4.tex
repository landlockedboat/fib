\section{\QnF actualitza diàriament les previsions de demanda per als pròxims 7 dies i fa una previsió molt acurada de les vendes del dia següent a cada botiga. A partir d’aquestes previsions quins processos o activitats s’haurien de planificar?}

Podem començar a planificar diversos processos segons de quina peça d'informació estem parlant.

\paragraph{Segons la informació referent a la demanda durant els pròxims set dies podem començar a planificar:}

\begin{description}
  \item[Proveïdors] Encarregar als proveïdors els productes necessaris per a la següent semana.
  \item[Producció] Planificar la distribució de treball i de processos que es duran a terme a l'obrador aquesta setmana.
\end{description}

\paragraph{Segons la previsió de vendes al dia següent per botiga podem planificar:}

\begin{description}
  \item[Personal] Redistribuïr el personal de vendes d'una botiga a una altra si el volum de vendes en una difereix dràsticament d'una altra (En una botiga s'esperen moltes vendes, en una altra no).
  \item[Logística] Generar l'horari de entregues de productes a cadascuna de les botigues.
  \item[Producció] Planejar l'horari i la distribució de tasques que l'obrador haurà de realitzar el dia següent, afinant el horari fet únicament amb la previsió setmanal.
\end{description}
