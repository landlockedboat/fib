\section{Si el govern de Pankilàndia es decidís a vendre Deute Públic 
estaria fent una política econòmica expansiva o restrictiva?}

Vendre deute públic produeix, en general, una etapa d'expansió
económica a curt termini, encara que a la llarga no és molt beneficiós
pel país, ja que perd una part del deute que hauria de cobrar
\cite{selling_debt}.

Tenint en compte que els seus deutors potser no són del tot fiables, 
seria una bona jugada per part del país la de vendre el deute.

En tot cas, es considera una mesura de política econòmica expansiva.
